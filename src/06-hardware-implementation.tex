\section{Hardware Implementation}
\label{sec:hardware}
Our second major contribution is the introduction of a generic, automatically synthesizable hardware implementation of the proposed approaches for the table-based function approximation.
\cref{fig:architecture} depicts this hardware architecture.\par
The architecture's input is a bit vector $\Xi$ to be evaluated by the approximation of $\Fx$.
The architecture's output is a bit vector $\Upsilon$.
The input and the output are assumed to be user-defined as fixed point numbers represented by the tuples $(S^{\Xi},W^{\Xi},F^{\Xi})$ and $(S^\Upsilon,W^\Upsilon,F^\Upsilon)$, respectively.
Here, $S$ indicates the sign, $W$ corresponds to the length of the binary bit string, and $F$ denotes the number of bits used for the fractional part.
First, the input $\Xi$ passes through an interval selector unit determining the sub-interval containing $\Xi$ and consequently, the values of parameters specific to the sub-interval e.g., the spacing between breakpoints.
Since the interval selector unit is implemented by using a comparator in each node of the binary tree generated from the set of sub-intervals, a single cycle implementation is not appropriate.
Moreover, the sequential segmentation approach even generates an unbalanced binary tree, resulting in a generally larger set of cascaded comparators than the other two segmentation approaches. 
In our design flow, a pre-processing balancing step is therefore applied for any set $\SetPartitions$ of intervals that always delivers a balanced binary tree of comparators.
Then, the address generator determines the addresses of the stored function values $A_i$ and $A_{i+1}$ corresponding to the  breakpoints enclosing the input.
These values are read from the \ac{BRAM}.
In the last stage, the subsequent block performs a linear interpolation to determine $\Upsilon$.\\
We implemented a design flow that automates the generation of the shown hardware architecture irrespective of the given function and interval.
First, one of the proposed interval splitting algorithms is selected and applied (see \cref{sec:proposed}).
The output $\SetPartitions$ of the algorithm is then used to generate a hardware description in VHDL.
For the determination of the range values $y_i$ to be stored in BRAMs, we employ the HDL coder of Matlab~\cite{MATLAB:2019} and adapt the code generation to instantiate \acp{BRAM}.
The set $\SetPartitions$ is also directly used to implement the interval selector and the linear interpolation blocks. 
The arithmetic operations performed to compute the output are pipelined to increase the throughput of the circuit. 
\begin{figure}[t!]
	\centering
	\resizebox{\textwidth}{!}{
		\begin{tikzpicture}
		\archTableBased
		\end{tikzpicture}
	}
	\caption{\label{fig:architecture} Proposed generic hardware implementation for table-based function approximation using interval splitting and BRAM instantiation. 
		An input $\Xi$ of $W^{\Xi}$ bits in pre-specified fixed-point number format is evaluated. 
		In just three clock cycles, the interval selector determines in which partition $\Xi$ is and its respective base address $A_i$ in the BRAM, as well as, a valid address is generated.
		In the next clock cycle, the two breakpoints required to evaluate $\Xi$ are looked up in the \ac{BRAM}. 
		Then, in another five clock cycles, the linear interpolation block calculates the approximated value of $\Fx$.
		The shown implementation is pipelined and has a latency of $L=9$ clock cycles.
	}
\end{figure}
The interval selector and address generator take three clock cycles together to generate valid address signals. 
The $y$ values are obtained from the BRAMs in the next clock cycle. 
Then, the pipelined linear interpolation block requiring five clock cycles produces the final output. 
Therefore, the latency ($L$) of the proposed architecture is constant at $L=9$ per function evaluation. 
Note that this latency is independent of the function to be approximated, number formats, and number of sub-intervals determined by interval splitting. 
In the following, we evaluate the implementation of our segmentation approach in a target \ac{FPGA} device to measure the memory footprint reductions, logic utilization (LUTs), BRAM utilization, and the achievable clock frequency.
