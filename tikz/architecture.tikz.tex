\tikzset{%
        brace/.style = { decorate, decoration={brace, amplitude=5pt} },
       mbrace/.style = { decorate, decoration={brace, amplitude=5pt, mirror} },
        label/.style = { black, midway, scale=0.5, align=center },
     toplabel/.style = { label, above=.5em, anchor=south },
    leftlabel/.style = { label,rotate=-90,left=.5em,anchor=north },   
  bottomlabel/.style = { label, below=.5em, anchor=north },
        force/.style = { rotate=-90,scale=0.4 },
        round/.style = { rounded corners=2mm },
       legend/.style = { right,scale=0.4 },
        nosep/.style = { inner sep=0pt },
   generation/.style = { anchor=base }
}
\makeatletter
\pgfdeclareshape{LUTShape}{
  % The 'minimum width' and 'minimum height' keys, not the content, determine
  % the size
  \savedanchor\northeast{%
    \pgfmathsetlength\pgf@x{\pgfshapeminwidth}%
    \pgfmathsetlength\pgf@y{\pgfshapeminheight}%
    \pgf@x=0.5\pgf@x
    \pgf@y=0.5\pgf@y
  }
  % This is redundant, but makes some things easier:
  \savedanchor\southwest{%
    \pgfmathsetlength\pgf@x{\pgfshapeminwidth}%
    \pgfmathsetlength\pgf@y{\pgfshapeminheight}%
    \pgf@x=-0.5\pgf@x
    \pgf@y=-0.5\pgf@y
  }
  % Inherit from rectangle
  \inheritanchorborder[from=rectangle]
  % Define same anchor a normal rectangle has
  \anchor{center}{\pgfpointorigin}
  \anchor{north}{\northeast \pgf@x=0pt}
  \anchor{east}{\northeast \pgf@y=0pt}
  \anchor{south}{\southwest \pgf@x=0pt}
  \anchor{west}{\southwest \pgf@y=0pt}
  \anchor{north east}{\northeast}
  \anchor{north west}{\northeast \pgf@x=-\pgf@x}
  \anchor{south west}{\southwest}
  \anchor{south east}{\southwest \pgf@x=-\pgf@x}
  \anchor{text}{
    \pgfpointorigin
    \advance\pgf@x by -.5\wd\pgfnodeparttextbox%
    \advance\pgf@y by -.5\ht\pgfnodeparttextbox%
    \advance\pgf@y by +.5\dp\pgfnodeparttextbox%
  }
  
 \anchor{CLK}{
    \pgf@process{\northeast}%
    \pgf@x=-1\pgf@x%
    \pgf@y=-.66666\pgf@y%
  }
 
  \anchor{outputLUT}{
    \pgf@process{\northeast}%
    \pgf@y=-0.1\pgf@y%
  }
  
  \anchor{inLUT}{
    \pgf@process{\northeast}%
    \pgf@x=-1\pgf@x%
    \pgf@y=-0.1\pgf@y%
  }
    
  % Draw the rectangle box and the port labels
  \backgroundpath{
    % Rectangle box
    \pgfpathrectanglecorners{\southwest}{\northeast}
    % Angle (>) for clock input
    \pgf@anchor@LUTShape@CLK
    \pgf@xa=\pgf@x \pgf@ya=\pgf@y
    \pgf@xb=\pgf@x \pgf@yb=\pgf@y
    \pgf@xc=\pgf@x \pgf@yc=\pgf@y
    \pgfmathsetlength\pgf@x{1.6ex} % size depends on font size
    \advance\pgf@ya by \pgf@x
    \advance\pgf@xb by \pgf@x
    \advance\pgf@yc by -\pgf@x
    \pgfpathmoveto{\pgfpoint{\pgf@xa}{\pgf@ya}}
    \pgfpathlineto{\pgfpoint{\pgf@xb}{\pgf@yb}}
    \pgfpathlineto{\pgfpoint{\pgf@xc}{\pgf@yc}}
    \pgfclosepath

    % Draw port labels
    \begingroup
    	
    	\pgf@anchor@LUTShape@outputLUT
    	\pgftext[bottom,at={\pgfpoint{\pgf@x}{\pgf@y}},y=\pgfshapeinnerysep]{\scriptsize }
        
        \pgf@anchor@LUTShape@inLUT
    	\pgftext[bottom,at={\pgfpoint{\pgf@x}{\pgf@y}},y=\pgfshapeinnerysep]{\scriptsize }
    \endgroup
  }
}

\newcommand{\archTableBased}
{

\tikzset{every LUTShape node/.style={draw,minimum width=1.3cm,minimum height=1.8cm,thick,inner sep=1mm,outer sep=0pt,cap=round}}

\tikzset{mux 6by3/.style={muxdemux, muxdemux def={Lh=6, NL=0, Rh=4, NB=0, NR=1,w=1, square pins=1}}}

\tikzset{mux 3by1/.style={muxdemux, muxdemux def={Lh=4, Rh=2,NL=3,NB=1, NR=1,w=2, square pins=1}}}

\node [mux 6by3,] (right) at (0,0) {};
\node [mux 6by3,rotate=180] (left) at (-2.5,0) {};

\draw[draw=black,thick] (left.north west) rectangle (right.north west);

\node (addGen) at (-5.0,0) [draw,thick,minimum width=1.5cm,minimum height=2.5cm] {\small \begin{tabular}{c} Address \\ Generator \end{tabular}};

\node (intSel) at (-7.55,0) [draw,thick,minimum width=1.5cm,minimum height=2.5cm] {\small \begin{tabular}{c} Interval \\ Selector \end{tabular}};

\node[shape=LUTShape] (registerOne) at (-10.2, 0){};

\node (BRAM) at (-1.2,0) {BRAM};

\node[shape=LUTShape] (registerThree) at (2.5, 3){};

\node[shape=LUTShape] (registerFour) at (2.5, -0.3){};

\node[shape=LUTShape] (registerFourF) at (2.5, -3.5){};

\node[shape=LUTShape] (registerFive) at (2.5, 5){};

\node[shape=LUTShape] (registerSix) at (2.5, -5.5){};


\node (linInter) at (5.5,-0.3) [draw,thick,minimum width=1.5cm,minimum height=4cm] {\small \begin{tabular}{c} Linear \\ Interpolation \end{tabular}};

\node[shape=LUTShape] (registerTwo) at (8.7, -0.3) {};

\draw (intSel.east) to[multiwire] (addGen.west);
\draw[<-,>=triangle 45] (addGen.west) to ($(addGen.west)-(0.2,0)$);
\draw[->,>=triangle 45] (addGen.east) to (left.rpin 1);

\draw ($(addGen.east)+(0.5,0)$) node[anchor=south] {$A_i$} to[short, *-] ($(addGen.east)+(0.5,-2.8)$) to ($(addGen.east)+(2.75,-2.8)$) node[fill=white,draw,thick,minimum width=1cm,minimum height=0.5cm,anchor=west] {$+1$} to ($(addGen.east)+(4.7,-2.8)$) |-  (right.rpin 1);
\draw ($(addGen.east)+(0.5,-2.8)$) to[multiwire] ($(addGen.east)+(2.75,-2.8)$);

\draw[->,>=triangle 45] (right.rpin 1) to ($(right.rpin 1)+(-0.2,0)$);

\draw ($(addGen.east)+(0.8,0)$) to[multiwire] ($(addGen.east)+(1.2,0)$);

%\draw[->,>=triangle 45] ($(BRAM)+(0,1.7)$) |- ($(registerThree.west)+(0,0.5)$);
\draw[->,>=triangle 45] (intSel.north) |- ($(registerThree.west)+(0,0)$);
\draw[] ($(registerThree.west)+(-7,0)$) to[multiwire=$\delta$,xshift=-0.3] (registerThree.west);

\draw[->,>=triangle 45] (intSel.south) |- ($(registerFourF.west)+(0,0)$);
\draw[] ($(registerFourF.west)+(-8,0)$) to[multiwire=$\frac{1}{\delta}$,xshift=-5cm] (registerFourF.west);

%\draw[->,>=triangle 45] (addGen.south) to ($(addGen.south)+(0,-2)$) to ($(addGen.south)+(6,-2)$)|- ($(registerFour.west)+(0,0)$);
\draw[->,>=triangle 45] ($(addGen.east)+(0.5,-2.8)$) node[anchor=south]{} to[short, *-] ($(addGen.east)+(0.5,-3.2)$) to ($(addGen.east)+(5,-3.2)$) |-  (registerFour.west);


%\draw($(registerThree.west)+(-2.8,0.5)$) to[multiwire=$W^\Upsilon$] ($(registerThree.west)+(0,0.5)$);

%\draw[->,>=triangle 45] ($(BRAM)+(0,-1.7)$) to ($(BRAM)+(0,-2.5)$) to ($(BRAM)+(2.3,-2.5)$) |- (registerFour.west);


\draw[->,>=triangle 45] (registerOne.east) to (intSel.west);
\draw[] (registerOne.east) to[multiwire=$W^{\Xi}$,xshift=-0.3] (intSel.west);

\draw[->,>=triangle 45] (registerOne.east) to ($(registerOne.east)+(0.2,0)$) to[short, *-] ($(registerOne.east)+(0.2,5)$) to ($(registerFive.west)+(0,0)$);

\draw[->,>=triangle 45] (registerFive.east) -|  ($(linInter.north)+(0.2,0)$);
\draw (registerFive.east) to[multiwire] ($(registerFive.east)+(2,0)$);

\draw[->,>=triangle 45] (registerThree.east) -|  ($(linInter.north)+(-0.2,0)$);
\draw (registerThree.east) to[multiwire] ($(registerThree.east)+(2,0)$);

\draw[->,>=triangle 45] (registerSix.east) -|  ($(linInter.south)+(0.2,0)$);
\draw (registerSix.east) to[multiwire] ($(registerSix.east)+(2,0)$);

\draw[->,>=triangle 45] (registerFourF.east) -|  ($(linInter.south)+(-0.2,0)$);
\draw (registerFourF.east) to[multiwire] ($(registerFourF.east)+(2,0)$);

\draw ($(registerOne.east)+(0.2,5)$) to[multiwire=$W^{\Xi}$] ($(registerFive.west)+(0,0)$);

\draw[->,>=triangle 45] (linInter.east) to (registerTwo.west);
\draw[] (linInter.east) to[multiwire=$W^\Upsilon$,xshift=-0.3] (registerTwo.west);

\draw[] (intSel.east) to ($(intSel.east)+(0.1,0)$) to [short, *-] ($(intSel.east)+(0.3,0)$);
\draw[->,>=triangle 45] ($(intSel.east)+(0.1,0)$)   |-  ($(registerSix.west)+(0,0)$);
\draw[] ($(registerSix.west)+(-5,0)$)   to[multiwire]  ($(registerSix.west)+(0,0)$);

\draw[->,>=triangle 45] ($(BRAM)+(0,1.7)$) to ($(BRAM)+(0,2.1)$) to ($(BRAM)+(2,2.1)$) |- ($(linInter.west)+(0,1.4)$);
\draw ($(linInter.west)+(-0.6,1.4)$) to[multiwire] ($(linInter.west)+(0,1.4)$);
\draw ($(BRAM)+(0,2.1)$) to[multiwire=$W^\Upsilon$] ($(BRAM)+(2,2.1)$);

\draw[->,>=triangle 45] ($(BRAM)+(0,-1.7)$) to ($(BRAM)+(0,-2.1)$)  to ($(BRAM)+(4.8,-2.1)$)  |- ($(linInter.west)+(0,-1.4)$);
\draw ($(linInter.west)+(-0.6,-1.4)$) to[multiwire] ($(linInter.west)+(0,-1.4)$);
\draw ($(BRAM)+(0.3,-2.1)$) to[multiwire=$W^\Upsilon$] ($(BRAM)+(2.6,-2.1)$);

\draw[<-,>=triangle 45] (registerOne.west) to ($(registerOne.west)+(-0.8,0)$) node[anchor=east] {$\Xi$};
\draw (registerOne.west) to[multiwire=$W^{\Xi}$] ($(registerOne.west)+(-0.8,0)$);

\draw[->,>=triangle 45] (registerFour) -- (linInter);
\draw (registerFour) to[multiwire] (linInter);

\draw ($(registerFour.west)+(-0.5,0)$) to[multiwire] (registerFour.west);

% clk signals
\draw (registerOne.CLK) to ($(registerOne.CLK)+(-0.3,0)$) to ($(registerOne.CLK)+(-0.3,-0.5)$) node[anchor=north] {clk};
\draw (registerTwo.CLK) to ($(registerTwo.CLK)+(-0.3,0)$) to ($(registerTwo.CLK)+(-0.3,-0.5)$) node[anchor=north] {clk};

\draw (registerThree.CLK) to ($(registerThree.CLK)+(-0.3,0)$) to ($(registerThree.CLK)+(-0.3,-0.5)$) node[anchor=north] {clk};
\draw (registerFour.CLK) to ($(registerFour.CLK)+(-0.3,0)$) to ($(registerFour.CLK)+(-0.3,-0.5)$) node[anchor=north] {clk};
\draw (registerFourF.CLK) to ($(registerFourF.CLK)+(-0.3,0)$) to ($(registerFourF.CLK)+(-0.3,-0.5)$) node[anchor=north] {clk};


\draw (registerFive.CLK) to ($(registerFive.CLK)+(-0.3,0)$) to ($(registerFive.CLK)+(-0.3,-0.5)$) node[anchor=north] {clk};

\draw (registerSix.CLK) to ($(registerSix.CLK)+(-0.3,0)$) to ($(registerSix.CLK)+(-0.3,-0.5)$) node[anchor=north] {clk};

\draw[->,>=triangle 45] (registerTwo.east) to ($(registerTwo.east)+(0.7,0)$) node[anchor=west] {$\Upsilon$};
\draw (registerTwo.east) to[multiwire=$W^\Upsilon$] ($(registerTwo.east)+(0.7,0)$);

\draw[->,>=triangle 45] (registerOne.east) to ($(registerOne.east)+(0.2,0)$) to[short, *-] ($(registerOne.east)+(0.2,-2.1)$) to ($(registerOne.east)+(3.1,-2.1)$) |- ($(addGen.west)+(0,-0.4)$);

\draw ($(registerOne.east)+(0.2,-2.1)$) to[multiwire=$W^\Xi$] ($(registerOne.east)+(2.9,-2.1)$);]


%\draw[line width=0.3mm,dashed] (-12.2,6.5) to (-12.2,-7.5);
\draw[line width=0.3mm,dashed] (-9.45,6.5) to (-9.45,-7.5);
\draw[line width=0.3mm,dashed] (-3.25,6.5) to (-3.25,-7.5);
\draw[line width=0.3mm,dashed] (3.5,6.5) to (3.5,-7.5);
\draw[line width=0.3mm,dashed] (7,6.5) to (7,-7.5);
%\draw[line width=0.3mm,dashed] (10.5,6.5) to (10.5,-7.5);

%\draw [decorate,decoration={brace,amplitude=10pt,mirror,raise=4pt},yshift=0pt] (-12.2,-7.5) -- (-9.5,-7.5) node [black,midway,xshift=0cm,yshift=-1cm] {1 clock cycle};
\draw [decorate,decoration={brace,amplitude=10pt,mirror,raise=4pt},yshift=0pt] (-9.45,-7.5) -- (-3.3,-7.5) node [black,midway,xshift=0cm,yshift=-1cm] {3 clock cycles};
\draw [decorate,decoration={brace,amplitude=10pt,mirror,raise=4pt},yshift=0pt] (-3.25,-7.5) -- (3.45,-7.5) node [black,midway,xshift=0cm,yshift=-1cm] {1 clock cycle};
\draw [decorate,decoration={brace,amplitude=10pt,mirror,raise=4pt},yshift=0pt] (3.5,-7.5) -- (6.9,-7.5) node [black,midway,xshift=0cm,yshift=-1cm] {5 clock cycles};
%\draw [decorate,decoration={brace,amplitude=10pt,mirror,raise=4pt},yshift=0pt] (7,-7.5) -- (10.5,-7.5) node [black,midway,xshift=0cm,yshift=-1cm] {1 clock cycle};

}
